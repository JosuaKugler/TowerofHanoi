\documentclass{beamer}
\usetheme{Warsaw}
\usepackage[ngerman]{babel}
\usepackage[utf8]{inputenc}
\usepackage[T1]{fontenc}
\usepackage{tikz,pgf}
\usepackage{pythontex}
\usepackage{comment}

\newtheorem{myexample}{Beispiel}
\renewcommand{\binom}[2]{\left( \genfrac{}{}{0pt}{}{ #1 }{ #2 } \right)}
\setbeamertemplate{footline}[frame number]

\title{Die Türme von Hanoi mit variabler Feldanzahl}
\author{BSNT 2019}
\date{\today}

\begin{document}
\frame{\titlepage}
\section{Einleitung}
\begin{frame}{Einleitung}
    \begin{block}{Türme von Hanoi}
    Es gibt $n$ Scheiben und $k$ Plätze, wobei zu Beginn alle Scheiben auf dem ersten Platz liegen.\\
    Ziel des Spieles ist es, alle Scheiben auf den letzte Platz zu stapeln.
    \begin{itemize}
        \item In jedem Zug wird die oberste Scheibe eines Platzes bewegt
        \item Es darf nie eine größere Scheibe auf eine kleinere Scheibe gelegt werden
    \end{itemize}
    \end{block}
\end{frame}
\subsection{Beispiele}
\pyc{from posterstuff import frame,gentikzincrement,nktable}
\pyc{frame(1,3,shownumber = True)}
\pyc{frame(2,3,shownumber = True)}
\pyc{frame(3,3,shownumber = True)}
\subsection{Hypothese für 3 Felder}
\begin{frame}{Gesetzmäßigkeit}
    \begin{definition}
    $M(n,k) := $ minimale Zugzahl bei $n$ Scheiben und $k$ Plätzen
    \end{definition}
    \begin{table}[h]
        \centering
        \begin{tabular}{c|c}
            Scheiben & Benötigte Züge \\
            1 & 1 \\
            2 & 3 \\
            3 & 7 \\
        \end{tabular}
        \caption{Zugzahl bei 3 Plätzen}
        \label{tab:3f}
    \end{table}
    \begin{alertblock}{Vermutung:}
    \[M(n,3) = 2^n-1\]
    \end{alertblock}
\end{frame}
\pyc{frame(4,3,0,7,shownumber = True)}
\begin{frame}{Was haben wir bisher gemacht?}
    \begin{itemize}
        \item 3 Scheiben vom ersten auf den zweiten Platz bewegt
        \item Benötigte Züge: 7
    \end{itemize}
    \begin{alertblock}{Interessant!}
    Im Prinzip ist das genau das, was wir bei 3 Scheiben gemacht haben:
    \begin{itemize}
        \item 3 Scheiben von einem Platz auf einen anderen bewegt
        \item Benötigte Züge: 7
    \end{itemize}
    \end{alertblock}
\end{frame}
\pyc{frame(4,3,7,15, shownumber = True)}

\subsection{Beweis der Hypothese}
\begin{frame}{Warum ist das so?}
    \begin{proof}
    \textbf{Induktionsannahme:} $M(n,3) = 2^n-1$\\
    \textbf{Induktionsschritt:}\\\ \\
        \begin{tabular}{l|l}
        Schritt & benötigte Zugzahl\\
        %\vspace{2mm}
        \hline
        Bewege $n$ Scheiben von Feld 1 auf Feld 2& $M(n,3) = 2^n-1$\\
        Bewege eine Scheibe von Feld 1 auf Feld 3&1\\
        Bewege $n$ Scheiben von Feld 2 auf Feld 3 & $M(n,3) = 2^n-1$\\
        &\\
        Gesamtes Umstapeln & $2^{n+1}-1$
    \end{tabular}
    \end{proof}
\end{frame}
\section{Variable Feldanzahl}
\subsection{Beispiel}
\pyc{frame(6,4)}

\subsection{Strategie}
\begin{frame}{Strategie}
    Wie gehen wir nun bei 4 oder mehr Plätzen vor?
    \begin{itemize}
        \item Wir stapeln einige Scheiben von Platz 1 auf Platz 2.
        \item Wir stapeln einige Scheiben von Platz 1 auf Platz 3.
        \item \dots
        \item Wir stapeln einige Scheiben von Platz 1 auf Platz $k-1$.
        \item Wir stapeln die größte Scheibe um.
        \item Wir stapeln die Scheiben von Platz $k-1$ auf Platz $k$.
        \item \dots
        \item Wir stapeln die Scheiben von Platz 3 auf Platz $k$.
        \item Wir stapeln die Scheiben von Platz 2 auf Platz $k$.
    \end{itemize}
\end{frame}
\subsection{Beweis}
\begin{frame}{Inkremente}
    \begin{itemize}
    \visible<1->{\item $I(n,k) := M(n,k)-M(n-1,k)$\\}
    \visible<2->{\item Wann ändert sich $I(n,k)$?\\}
    \visible<3->{\item Genau dann, wenn sich das Inkrement eines Zwischenturms ändert}
    \end{itemize}
    \visible<4->{\pyc{gentikzincrement([8,5,{1:[0],2:[6,7], 3:[3,4,5], 4:[1,2]}])}}
\end{frame}
\begin{frame}{Inkrementblöcke}
    \begin{itemize}
        \item<1-> rekursive Formel für die Inkrementblocklänge?
        \item<2-> Die Inkrementblocklänge bei $m$ Feldern und Inkrement $I$ entspricht der Summe aller Inkrementblocklängen für das Inkrement $\frac{I}{2}$ und einer Felderzahl $\leq m$
        \item<3-> Anfangsbedingungen:
        \begin{itemize}
            \item<4-> Inkrementblocklänge bei 3 Feldern ist stets 1
            \item<5-> Inkrementblocklänge für das Inkrement 1 ist stets 1
        \end{itemize}
    \end{itemize}
\end{frame}
\input{slides/tables.tex}
\begin{frame}{Formel}
    Die Länge eines $t$-Blocks beträgt im Allgemeinen \[\binom{t+k-3}{k-3}\]
    Die Zugzahl entspricht der Summe aller Inkrementblocklängen gewichtet mit ihrem Inkrement.
    \[M(n,k)=\sum_{i=0}^{t}\left(2^i*\binom{i+k-3}{k-3}\right)\]
\end{frame}
\begin{frame}{Formel}
    Für gegebene $n,k$ müssen wir nun das Inkrement $2^t$ bestimmen.
    Es gilt \[\binom{t+k-2}{k-2}\geq n>\binom{t+k-3}{k-2}\]
    Bei nicht abgeschlossenem Inkrementblock muss zudem noch etwas abgezogen werden.
    \[2^t\left(\binom{t+k-2}{k-2}-n\right)\]
\end{frame}
\begin{frame}{Formel}
    Sei $n$ die Anzahl der Scheiben, $k$ die Anzahl der Felder und $t$ eine natürliche Zahl, sodass
    \[\binom{t+k-2}{k-2}\geq n>\binom{t+k-3}{k-2}\]
    Dann gilt
    \[M(n,k)=2^t\left(n-\binom{t+k-2}{k-2}\right)+\sum_{i=0}^{t}\left(2^i*\binom{i+k-3}{k-3}\right)\]
\end{frame}
\section{Anzahl der Möglichkeiten}
\subsection{Rekursive Formel}
\begin{frame}{Verschiedene Zwischenturmkonfigurationen}
    \scalebox{0.5}{
        \pyc{gentikzincrement([11,5,{1:[0],2:[6,7,8,9,10,11], 3:[3,4,5], 4:[1,2]}])}
        \pyc{gentikzincrement([11,5,{1:[0],2:[8,9,10,11], 3:[3,4,5,6,7], 4:[1,2]}])}}
    \scalebox{0.5}{   
        \pyc{gentikzincrement([11,5,{1:[0],2:[7,8,9,10,11], 3:[3,4,5,6], 4:[1,2]}])}
        \pyc{gentikzincrement([11,5,{1:[0],2:[7,8,9,10,11], 3:[4,5,6], 4:[1,2,3]}])}}
    \scalebox{0.5}{
        \pyc{gentikzincrement([11,5,{1:[0],2:[8,9,10,11], 3:[4,5,6,7], 4:[1,2,3]}])}}
\end{frame}
\begin{frame}{Rekursive Darstellung}
    \begin{itemize}
        \item Höhe $h_1$ des ersten Zwischenturms innerhalb des \glqq grünen Bereichs\grqq\ variieren
        \item Anzahl der Möglichkeiten für ein bestimmtes $h_1$
        \begin{itemize}
            \item Anzahl der Möglichkeiten, den ersten Zwischenturm zu errichten
            \item mal Anzahl der Möglichkeiten, die restliche Konfiguration zu errichten
        \end{itemize}
        \item Für alle $h_1$ aufsummieren
    \end{itemize}
\end{frame}
\begin{frame}{Rekursive Darstellung}
    Sei $\Upsilon(n,k)$ die Anzahl der Möglichkeiten bei $n$ Scheiben und $k$ Feldern. Dann gilt
    \[\Upsilon(n,k) = \sum_{h_1 = h_{1,\mathrm{min}}}^{h_{1, \mathrm{max}}}\Upsilon(h_1,k)\Upsilon(n-h_1,k-1)\]
\end{frame}
\begin{frame}{Rekursive Darstellung}
\begin{itemize}
    \item Die Grenzen für $h_{1,\mathrm{min}}$ und  $h_{1, \mathrm{max}}$ liegen eigentlich bei den Grenzen des Inkrementblocks.
    \item Allerdings gibt es noch weitere Einschränkungen für $h_{1,\mathrm{min}}$ und  $h_{1, \mathrm{max}}$.
\end{itemize}
\ \\
\begin{tabular}{p{0.5\textwidth}p{0.5\textwidth}}
    Fall 1&Fall 2
\end{tabular}
\scalebox{0.5}{
    \pyc{gentikzincrement([15,5,{1:[0],2:[10,11,12,13,14], 3:[4,5,6,7,8,9], 4:[1,2,3]}])}
    \pyc{gentikzincrement([15,5,{1:[0],2:[6,7,8,9,10,11,12,13,14], 3:[3,4,5], 4:[1,2]}])}}
\end{frame}
\begin{frame}{Rekursive Darstellung}
    \[\Upsilon(1,k)=1\]
    \[\Upsilon(n,3) = 1\]
    {\small \begin{align*}
        \Upsilon(n,k) 
        =\sum  _{h_1 = \binom{\tau+k-4}{k-2}+\mathrm{max} \left(0;\ n-\binom{\tau+k-3}{\tau-1} - \binom{\tau+k-4}{\tau} \right)}
                ^{\binom{\tau+k-4}{k-2}+\mathrm{min}\left(\binom{\tau+k-4}{\tau-1};\ n-\binom{\tau+k-3}{\tau-1}\right)}
            \left(
                \Upsilon(h_1,k) \cdot \Upsilon(n-h_1,k-1) 
            \right)
    \end{align*}
    }
\end{frame}
\subsection{Vermutung explizite Darstellung}
\input{slides/upsilon_table.tex}
\begin{frame}{Vermutung}
    Sei $n$ die Anzahl der Scheiben, $k$ die Anzahl der Felder und $t$ eine natürliche Zahl mit
    \[\binom{t+k-2}{k-2}\geq n>\binom{t+k-3}{k-2}\]
    Dann ist \[\Upsilon(n,k) = \binom{\binom{t+k-3}{t}}{\binom{t+k-2}{t}-n}\]
\end{frame}
\begin{frame}
    \Huge Vielen Dank für eure Aufmerksamkeit
\end{frame}
\end{document}