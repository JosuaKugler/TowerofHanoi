\begin{frame}{Warum ist das so?}
    \begin{proof}
    \textbf{Induktionsanfang: ($n=1$)} $M(1,3) = 2^1-1 = 1$
    \textbf{Induktionsannahme: ($n=k$)} $M(k,3) = 2^k-1$
    \textbf{Induktionsschritt: ($n= k+1$)}
    \begin{itemize}
        \item Bewege $k$ Scheiben von Feld 1 auf Feld 2
        \item Bewege die größte Scheibe von Feld 1 auf Feld 3
        \item Bewege $k$ Scheiben von Feld 2 auf Feld 3
    \end{itemize}
    Das benötigt insgesamt $M(k,3)+1+M(k,3)$ Züge. Wir können aufgrund der Induktionsannahme $M(k,3) = 2^k-1$ annehmen.
    \[M(k+1,3) = 2*M(k,3)+1 = 2*(2^k-1)+1 = 2^{k+1}-1\]
    \end{proof}
\end{frame}