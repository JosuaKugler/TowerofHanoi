\documentclass{beamer}
\usetheme{Warsaw}
\usepackage[ngerman]{babel}
\usepackage[utf8]{inputenc}
\usepackage[T1]{fontenc}
\usepackage{tikz,pgf}
\usepackage{pythontex}
\usepackage{comment}

\newtheorem{myexample}{Beispiel}
\renewcommand{\binom}[2]{\left( \genfrac{}{}{0pt}{}{ #1 }{ #2 } \right)}

\title{Die Türme von Hanoi mit variabler Platzanzahl}
\author{Heidelberger Schülersymposium 2019}
\date{13.05.2019}

\begin{document}
\frame{\titlepage}
\section{Einleitung}
\begin{frame}{Einleitung}
    \begin{block}{Türme von Hanoi}
    Es gibt $n$ Scheiben und $k$ Plätze, wobei zu Beginn alle Scheiben auf dem ersten Platz liegen.\\
    Ziel des Spieles ist es, alle Scheiben auf den letzte Platz zu stapeln.
    \begin{itemize}
        \item In jedem Zug wird die oberste Scheibe eines Platzes bewegt
        \item Es darf nie eine größere Scheibe auf eine kleinere Scheibe gelegt werden
    \end{itemize}
    \end{block}
\end{frame}
\section{3 Plätze}
\subsection{Beispiele}
\pyc{from genlatexrainbowdevelop import frames}
\pyc{frames(1,3)}
\pyc{frames(2,3)}
\pyc{frames(3,3)}
\subsection{Hypothese für 3 Plätze}
\begin{frame}{Gesetzmäßigkeit}
    \begin{definition}
    $M(n,k) := $ minimale Zugzahl bei $n$ Scheiben und $k$ Plätzen
    \end{definition}
    \begin{table}[h]
        \centering
        \begin{tabular}{c|c}
            Scheiben & Benötigte Züge \\
            1 & 1 \\
            2 & 3 \\
            3 & 7 \\
        \end{tabular}
        \caption{Zugzahl bei 3 Plätzen}
        \label{tab:3f}
    \end{table}
    \begin{alertblock}{Vermutung:}
    \[M(n,3) = 2^n-1\]
    \end{alertblock}
\end{frame}
\subsection{4 Scheiben}
\pyc{frames(4,3,0,7)}
\begin{frame}{Was haben wir bisher gemacht?}
    \begin{itemize}
        \item 3 Scheiben vom ersten auf den zweiten Platz bewegt
        \item Benötigte Züge: 7
    \end{itemize}
    \begin{alertblock}{Interessant!}
    Im Prinzip ist das genau das, was wir bei 3 Scheiben gemacht haben:
    \begin{itemize}
        \item 3 Scheiben von einem Platz auf einen anderen bewegt
        \item Benötigte Züge: 7
    \end{itemize}
    \end{alertblock}
\end{frame}
\pyc{frames(4,3,8,15)}
\subsection{Beweis der Hypothese}
\begin{frame}{Warum ist das so?}
    \begin{proof}
    \textbf{Induktionsanfang: ($n=1$)} $M(1,3) = 2^1-1 = 1$
    \textbf{Induktionsannahme: ($n=k$)} $M(k,3) = 2^k-1$
    \textbf{Induktionsschritt: ($n= k+1$)}
    \begin{itemize}
        \item Bewege $k$ Scheiben von Feld 1 auf Feld 2
        \item Bewege die größte Scheibe von Feld 1 auf Feld 3
        \item Bewege $k$ Scheiben von Feld 2 auf Feld 3
    \end{itemize}
    Das benötigt insgesamt $M(k,3)+1+M(k,3)$ Züge. Wir können aufgrund der Induktionsannahme $M(k,3) = 2^k-1$ annehmen.
    \[M(k+1,3) = 2*M(k,3)+1 = 2*(2^k-1)+1 = 2^{k+1}-1\]
    \end{proof}
\end{frame}
\section{Variable Platzanzahl}
\subsection{6 Scheiben bei 4 Plätzen}
\pyc{frames(6,4)}
\subsection{Strategie}
\begin{frame}{Strategie}
    Wie gehen wir nun bei 4 oder mehr Plätzen vor?
    \begin{itemize}
        \item Wir stapeln einige Scheiben von Platz 1 auf Platz 2.
        \item Wir stapeln einige Scheiben von Platz 1 auf Platz 3.
        \item \dots
        \item Wir stapeln einige Scheiben von Platz 1 auf Platz $k-1$.
        \item Wir stapeln die größte Scheibe um.
        \item Wir stapeln die Scheiben von Platz $k-1$ auf Platz $k$.
        \item \dots
        \item Wir stapeln die Scheiben von Platz 3 auf Platz $k$.
        \item Wir stapeln die Scheiben von Platz 2 auf Platz $k$.
    \end{itemize}
\end{frame}
\subsection{Fragen}
%\begin{frame}{Fragen}
    \begin{enumerate}
    \item Wenn man diese Anzahlen möglichst gut wählt, ist diese Strategie dann überhaupt optimal?
    \item Wie viele Züge braucht man denn jetzt mit dieser Strategie?
    \item Wie muss man die Anzahlen überhaupt wählen, damit man möglichst wenige Züge braucht?
    \item Wie viele Möglichkeiten gibt es, eine Startkonfiguration zu lösen, indem man diese Anzahlen variiert?
    \end{enumerate}
\end{frame}
\subsection{Antworten}
%\begin{frame}{Wenn man diese Anzahlen möglichst gut wählt, ist diese Strategie dann überhaupt optimal?}
    \begin{itemize}
        \item Ja
        \item Es wurde 70 Jahre lang nach diesem Beweis gesucht, wir werden ihn also nicht \glqq kurz \grqq erklären
    \end{itemize}
\end{frame}
%\begin{frame}{Wie viele Züge braucht man denn jetzt mit dieser Strategie?}
    Sei $n$ die Anzahl der Scheiben, $k$ die Anzahl der Felder und $t$ eine natürliche Zahl, sodass
    \[\binom{t+k-2}{k-2}\geq n>\binom{t+k-3}{k-2}\]
    Dann gilt
    \[M(n,k)=\sum_{i=0}^{t}\left(2^i*\binom{i+k-3}{k-3}\right)+2^t\left(n-\binom{t+k-2}{k-2}\right)\]
\end{frame}
%\begin{frame}{Wie muss man die Anzahlen überhaupt wählen, damit man möglichst wenige Züge braucht?}
    Sei $n$ die Anzahl der Scheiben, $k$ die Anzahl der Plätze und $p$ der Platz, auf welchen die Scheiben gelegt werden.
    Dann gilt für die Anzahl der Scheiben $N(n,k,p)$:
    \[N(n,k,p)=\binom{t+k-2-p}{t-1}\]
\end{frame}
%\begin{frame}{Wie viele Möglichkeiten gibt es, eine Startkonfiguration zu lösen, indem man diese Anzahlen variiert?}
    Sei $n$ die Anzahl der Scheiben, $k$ die Anzahl der Plätze und $t$ eine natürliche Zahl, sodass
    \[\binom{t+k-2}{k-2}\geq n>\binom{t+k-3}{k-2}\]
    Dann gilt für die Anzahl der Möglichkeiten $\Upsilon(n,k,t)$:
    \[\Upsilon(n,k,t)=\binom{\binom{t+k-3}{t}}{\binom{t+k-2}{t}-n}\]
\end{frame}
\section{Kryptographie}
\begin{frame}{Kryptographie}
    \begin{itemize}
        \item Wir verschlüsseln den Text \glqq Langer Text \grqq
        \item Dazu wählen wir 4 Parameter: $n$, $k$, \textit{step} und \textit{movecount}
    \end{itemize}
    \begin{myexample}
        $n=7$, $k=4$, \textit{step} = 4, \textit{movecount} = 15
    \end{myexample}
\end{frame}
\pyc{frames(7,4,begin = 0, end = 15,text = "Langer ")}
\begin{frame}
        \begin{itemize}
            \item Wir verschlüsseln den Text \glqq Langer Text\grqq
            \item Dazu wählen wir 4 Parameter: $n$, $k$, \textit{step} und \textit{movecount}
            \item Dann verschlüsseln wir die ersten $n$ Zeichen, indem wir eine Konfiguration mit $k$ Plätzen \textit{movecount} Züge weit lösen
        \end{itemize}
    \begin{myexample}
        \begin{itemize}
            \item $n=7$, $k=4$, \textit{step} = 4, \textit{movecount} = 15
            \item Nach der ersten Runde erhalten wir folgenden Text:\\ \glqq ner aLg\grqq + \glqq Text\grqq = \glqq ner aLgText\grqq
            \item Nun gehen wir in diesem Text \textit{step} nach rechts und führen die gleiche Prozedur wieder durch: \glqq ner \textcolor{red}{aLgText}\grqq
        \end{itemize}   
    \end{myexample}
\end{frame}
\pyc{frames(7,4,begin = 0, end = 15,text = "aLgText")}
\begin{frame}
        \begin{itemize}
            \item Wir verschlüsseln den Text \glqq Langer Text\grqq
            \item Dazu wählen wir 4 Parameter: $n$, $k$, \textit{step} und \textit{movecount}
            \item Dann verschlüsseln wir die ersten $n$ Zeichen, indem wir eine Konfiguration mit $k$ Plätzen \textit{movecount} Züge weit lösen
            \item Nun nehmen wir ab dem \textit{step}+1-ten Zeichen wieder $n$ Zeichen und lösen eine Konfiguration mit $k$ Plätzen \textit{movecount} Züge weit
        \end{itemize}
    \begin{myexample}
        \begin{itemize}
            \item $n=7$, $k=4$, \textit{step} = 4, \textit{movecount} = 15
            \item Nach der zweiten Runde erhalten wir folgenden Text: \glqq ner \grqq + \glqq gextLaT\grqq = \glqq ner gextLaT\grqq
        \end{itemize}
    \end{myexample}
\end{frame}
\begin{frame}
    \Huge Vielen Dank für eure Aufmerksamkeit
\end{frame}
%\pyc{frames(7,4,begin = 0, end = 15,text = "LaTner ")}
\end{document}